\documentclass{article}
\usepackage{amsmath}
\usepackage{geometry}
\usepackage{multicol}
\geometry{a4paper, margin=1in}

\begin{document}

\title{Derivation of the Inverse CDF of Different Distributions}
\author{}
\date{}
\maketitle

\section*{Exponential Distribution}

\subsection*{CDF:}
The cumulative distribution function (CDF) of an exponential distribution is given by:
\[
F(x) = 1 - e^{-\lambda x}, \quad x \geq 0
\]

\subsection*{Inverse CDF:}
To find the inverse CDF, we set \( F(x) = u \):
\[
u = 1 - e^{-\lambda x}
\]

Solving for \( x \):
\[
e^{-\lambda x} = 1 - u
\]

Taking the natural logarithm of both sides:
\[
-\lambda x = \ln(1 - u)
\]

Solving for \( x \):
\[
x = -\frac{1}{\lambda} \ln(1 - u)
\]

Thus, given a uniform random variable \( U \sim \text{Uniform}(0, 1) \), the sample from the exponential distribution is:
\[
X = -\frac{1}{\lambda} \ln(1 - U)
\]

\newpage

\section*{Weibull Distribution}

\subsection*{CDF:}
The cumulative distribution function (CDF) of the Weibull distribution is given by:
\[
F(x) = 1 - e^{-(x/\lambda)^k}, \quad x \geq 0
\]

\subsection*{Inverse CDF:}
To find the inverse CDF, we set \( F(x) = u \):
\[
u = 1 - e^{-(x/\lambda)^k}
\]

Solving for \( x \):
\[
e^{-(x/\lambda)^k} = 1 - u
\]

Taking the natural logarithm of both sides:
\[
-(x/\lambda)^k = \ln(1 - u)
\]

Solving for \( (x/\lambda)^k \):
\[
(x/\lambda)^k = -\ln(1 - u)
\]

Taking the \( k \)-th root of both sides:
\[
x = \lambda (-\ln(1 - u))^{1/k}
\]

Thus, given a uniform random variable \( U \sim \text{Uniform}(0, 1) \), the sample from the Weibull distribution is:
\[
X = \lambda (-\ln(1 - U))^{1/k}
\]

\newpage

\section*{Logistic Distribution}

\subsection*{CDF:}
The cumulative distribution function (CDF) of the logistic distribution is given by:
\[
F(x) = \frac{1}{1 + e^{-(x-\mu)/s}}
\]

\subsection*{Inverse CDF:}
To find the inverse CDF, we set \( F(x) = u \):
\[
u = \frac{1}{1 + e^{-(x-\mu)/s}}
\]

Solving for \( x \):
\[
\frac{1}{1 + e^{-(x-\mu)/s}} = \frac{1}{u}
\]

\[
1 + e^{-(x-\mu)/s} = \frac{1}{u}
\]

\[
e^{-(x-\mu)/s} = \frac{1}{u} - 1
\]

\[
\frac{1}{e^{(x-\mu)/s}} = \frac{u}{1 - u}
\]

\[
-(x-\mu)/s = \ln\left(\frac{u}{1 - u}\right)
\]

\[
x - \mu = -s \ln\left(\frac{u}{1 - u}\right)
\]

\[
x = \mu - s \ln\left(\frac{u}{1 - u}\right)
\]

Thus, given a uniform random variable \( U \sim \text{Uniform}(0, 1) \), the sample from the logistic distribution is:
\[
X = \mu - s \ln\left(\frac{U}{1 - U}\right)
\]

\newpage

\section*{Geometric Distribution}

\subsection*{CDF:}
The cumulative distribution function (CDF) of the geometric distribution is given by:
\[
F(k) = 1 - (1 - p)^k, \quad k = 0, 1, 2, \ldots
\]

\subsection*{Inverse CDF:}
To find the inverse CDF, we set \( F(k) = u \):
\[
u = 1 - (1 - p)^k
\]

\[
1 - u = (1 - p)^k
\]

\[
\ln(1 - u) = k \ln(1 - p)
\]

\[
k = \frac{\ln(1 - u)}{\ln(1 - p)}
\]

Since \(k\) must be an integer, we take the ceiling of this value:
\[
k = \lceil \frac{\ln(1 - u)}{\ln(1 - p)} \rceil
\]

Thus, given a uniform random variable \( U \sim \text{Uniform}(0, 1) \), the sample from the geometric distribution is:
\[
X = \lceil \frac{\ln(1 - U)}{\ln(1 - p)} \rceil
\]

\end{document}
